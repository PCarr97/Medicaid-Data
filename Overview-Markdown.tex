% Options for packages loaded elsewhere
\PassOptionsToPackage{unicode}{hyperref}
\PassOptionsToPackage{hyphens}{url}
%
\documentclass[
]{article}
\usepackage{amsmath,amssymb}
\usepackage{lmodern}
\usepackage{iftex}
\ifPDFTeX
  \usepackage[T1]{fontenc}
  \usepackage[utf8]{inputenc}
  \usepackage{textcomp} % provide euro and other symbols
\else % if luatex or xetex
  \usepackage{unicode-math}
  \defaultfontfeatures{Scale=MatchLowercase}
  \defaultfontfeatures[\rmfamily]{Ligatures=TeX,Scale=1}
\fi
% Use upquote if available, for straight quotes in verbatim environments
\IfFileExists{upquote.sty}{\usepackage{upquote}}{}
\IfFileExists{microtype.sty}{% use microtype if available
  \usepackage[]{microtype}
  \UseMicrotypeSet[protrusion]{basicmath} % disable protrusion for tt fonts
}{}
\makeatletter
\@ifundefined{KOMAClassName}{% if non-KOMA class
  \IfFileExists{parskip.sty}{%
    \usepackage{parskip}
  }{% else
    \setlength{\parindent}{0pt}
    \setlength{\parskip}{6pt plus 2pt minus 1pt}}
}{% if KOMA class
  \KOMAoptions{parskip=half}}
\makeatother
\usepackage{xcolor}
\usepackage[margin=1in]{geometry}
\usepackage{color}
\usepackage{fancyvrb}
\newcommand{\VerbBar}{|}
\newcommand{\VERB}{\Verb[commandchars=\\\{\}]}
\DefineVerbatimEnvironment{Highlighting}{Verbatim}{commandchars=\\\{\}}
% Add ',fontsize=\small' for more characters per line
\usepackage{framed}
\definecolor{shadecolor}{RGB}{248,248,248}
\newenvironment{Shaded}{\begin{snugshade}}{\end{snugshade}}
\newcommand{\AlertTok}[1]{\textcolor[rgb]{0.94,0.16,0.16}{#1}}
\newcommand{\AnnotationTok}[1]{\textcolor[rgb]{0.56,0.35,0.01}{\textbf{\textit{#1}}}}
\newcommand{\AttributeTok}[1]{\textcolor[rgb]{0.77,0.63,0.00}{#1}}
\newcommand{\BaseNTok}[1]{\textcolor[rgb]{0.00,0.00,0.81}{#1}}
\newcommand{\BuiltInTok}[1]{#1}
\newcommand{\CharTok}[1]{\textcolor[rgb]{0.31,0.60,0.02}{#1}}
\newcommand{\CommentTok}[1]{\textcolor[rgb]{0.56,0.35,0.01}{\textit{#1}}}
\newcommand{\CommentVarTok}[1]{\textcolor[rgb]{0.56,0.35,0.01}{\textbf{\textit{#1}}}}
\newcommand{\ConstantTok}[1]{\textcolor[rgb]{0.00,0.00,0.00}{#1}}
\newcommand{\ControlFlowTok}[1]{\textcolor[rgb]{0.13,0.29,0.53}{\textbf{#1}}}
\newcommand{\DataTypeTok}[1]{\textcolor[rgb]{0.13,0.29,0.53}{#1}}
\newcommand{\DecValTok}[1]{\textcolor[rgb]{0.00,0.00,0.81}{#1}}
\newcommand{\DocumentationTok}[1]{\textcolor[rgb]{0.56,0.35,0.01}{\textbf{\textit{#1}}}}
\newcommand{\ErrorTok}[1]{\textcolor[rgb]{0.64,0.00,0.00}{\textbf{#1}}}
\newcommand{\ExtensionTok}[1]{#1}
\newcommand{\FloatTok}[1]{\textcolor[rgb]{0.00,0.00,0.81}{#1}}
\newcommand{\FunctionTok}[1]{\textcolor[rgb]{0.00,0.00,0.00}{#1}}
\newcommand{\ImportTok}[1]{#1}
\newcommand{\InformationTok}[1]{\textcolor[rgb]{0.56,0.35,0.01}{\textbf{\textit{#1}}}}
\newcommand{\KeywordTok}[1]{\textcolor[rgb]{0.13,0.29,0.53}{\textbf{#1}}}
\newcommand{\NormalTok}[1]{#1}
\newcommand{\OperatorTok}[1]{\textcolor[rgb]{0.81,0.36,0.00}{\textbf{#1}}}
\newcommand{\OtherTok}[1]{\textcolor[rgb]{0.56,0.35,0.01}{#1}}
\newcommand{\PreprocessorTok}[1]{\textcolor[rgb]{0.56,0.35,0.01}{\textit{#1}}}
\newcommand{\RegionMarkerTok}[1]{#1}
\newcommand{\SpecialCharTok}[1]{\textcolor[rgb]{0.00,0.00,0.00}{#1}}
\newcommand{\SpecialStringTok}[1]{\textcolor[rgb]{0.31,0.60,0.02}{#1}}
\newcommand{\StringTok}[1]{\textcolor[rgb]{0.31,0.60,0.02}{#1}}
\newcommand{\VariableTok}[1]{\textcolor[rgb]{0.00,0.00,0.00}{#1}}
\newcommand{\VerbatimStringTok}[1]{\textcolor[rgb]{0.31,0.60,0.02}{#1}}
\newcommand{\WarningTok}[1]{\textcolor[rgb]{0.56,0.35,0.01}{\textbf{\textit{#1}}}}
\usepackage{graphicx}
\makeatletter
\def\maxwidth{\ifdim\Gin@nat@width>\linewidth\linewidth\else\Gin@nat@width\fi}
\def\maxheight{\ifdim\Gin@nat@height>\textheight\textheight\else\Gin@nat@height\fi}
\makeatother
% Scale images if necessary, so that they will not overflow the page
% margins by default, and it is still possible to overwrite the defaults
% using explicit options in \includegraphics[width, height, ...]{}
\setkeys{Gin}{width=\maxwidth,height=\maxheight,keepaspectratio}
% Set default figure placement to htbp
\makeatletter
\def\fps@figure{htbp}
\makeatother
\setlength{\emergencystretch}{3em} % prevent overfull lines
\providecommand{\tightlist}{%
  \setlength{\itemsep}{0pt}\setlength{\parskip}{0pt}}
\setcounter{secnumdepth}{-\maxdimen} % remove section numbering
\ifLuaTeX
  \usepackage{selnolig}  % disable illegal ligatures
\fi
\IfFileExists{bookmark.sty}{\usepackage{bookmark}}{\usepackage{hyperref}}
\IfFileExists{xurl.sty}{\usepackage{xurl}}{} % add URL line breaks if available
\urlstyle{same} % disable monospaced font for URLs
\hypersetup{
  pdftitle={Medicaid Data Exploration},
  pdfauthor={Patrick Carr},
  hidelinks,
  pdfcreator={LaTeX via pandoc}}

\title{Medicaid Data Exploration}
\author{Patrick Carr}
\date{02-07-2023}

\begin{document}
\maketitle

\hypertarget{data-exploration}{%
\subsection{Data Exploration}\label{data-exploration}}

This exploration will look at the impact of the increase in Medicaid
eligibility as a percentage of the federal poverty line.

I will be exploring trends in five separate data sets. First, I will
look at the number of emergency room visits per thousand Americans, pear
year:

\begin{Shaded}
\begin{Highlighting}[]
\FunctionTok{install.packages}\NormalTok{(}\StringTok{\textquotesingle{}pacman\textquotesingle{}}\NormalTok{, }\AttributeTok{repos =} \StringTok{"http://cran.us.r{-}project.org"}\NormalTok{)}
\end{Highlighting}
\end{Shaded}

\begin{verbatim}
## Installing package into 'C:/Users/Patrick Carr/Documents/R/win-library/4.1'
## (as 'lib' is unspecified)
\end{verbatim}

\begin{verbatim}
## package 'pacman' successfully unpacked and MD5 sums checked
## 
## The downloaded binary packages are in
##  C:\Users\Patrick Carr\AppData\Local\Temp\RtmpMNDQHX\downloaded_packages
\end{verbatim}

\begin{Shaded}
\begin{Highlighting}[]
\FunctionTok{library}\NormalTok{(pacman)}
\FunctionTok{p\_load}\NormalTok{(tidyverse, janitor, lubridate, dplyr, ggplot, tinytex)}
\end{Highlighting}
\end{Shaded}

\begin{verbatim}
## Installing package into 'C:/Users/Patrick Carr/Documents/R/win-library/4.1'
## (as 'lib' is unspecified)
\end{verbatim}

\begin{verbatim}
## Warning: package 'ggplot' is not available for this version of R
## 
## A version of this package for your version of R might be available elsewhere,
## see the ideas at
## https://cran.r-project.org/doc/manuals/r-patched/R-admin.html#Installing-packages
\end{verbatim}

\begin{verbatim}
## Warning: unable to access index for repository http://www.stats.ox.ac.uk/pub/RWin/bin/windows/contrib/4.2:
##   cannot open URL 'http://www.stats.ox.ac.uk/pub/RWin/bin/windows/contrib/4.2/PACKAGES'
\end{verbatim}

\begin{verbatim}
## Warning: 'BiocManager' not available.  Could not check Bioconductor.
## 
## Please use `install.packages('BiocManager')` and then retry.
\end{verbatim}

\begin{verbatim}
## Warning in p_install(package, character.only = TRUE, ...):
\end{verbatim}

\begin{verbatim}
## Warning in library(package, lib.loc = lib.loc, character.only = TRUE,
## logical.return = TRUE, : there is no package called 'ggplot'
\end{verbatim}

\begin{verbatim}
## Warning in p_load(tidyverse, janitor, lubridate, dplyr, ggplot, tinytex): Failed to install/load:
## ggplot
\end{verbatim}

\begin{Shaded}
\begin{Highlighting}[]
\CommentTok{\# Import and format data on Medicaid eligibility cutoffs}
\NormalTok{eligibility }\OtherTok{\textless{}{-}} \FunctionTok{read\_csv}\NormalTok{(}\StringTok{"Raw Data/eligibility.csv"}\NormalTok{, }
                        \AttributeTok{col\_names =} \FunctionTok{c}\NormalTok{(}\StringTok{"Year"}\NormalTok{, }\StringTok{"Percent"}\NormalTok{), }\AttributeTok{skip =} \DecValTok{2}\NormalTok{)}
\end{Highlighting}
\end{Shaded}

\begin{verbatim}
## Rows: 20 Columns: 2
\end{verbatim}

\begin{verbatim}
## -- Column specification --------------------------------------------------------
## Delimiter: ","
## dbl (2): Year, Percent
## 
## i Use `spec()` to retrieve the full column specification for this data.
## i Specify the column types or set `show_col_types = FALSE` to quiet this message.
\end{verbatim}

\begin{Shaded}
\begin{Highlighting}[]
\NormalTok{eligibility}\SpecialCharTok{$}\NormalTok{Year }\OtherTok{\textless{}{-}} \FunctionTok{as.Date}\NormalTok{(}\FunctionTok{as.character}\NormalTok{(eligibility}\SpecialCharTok{$}\NormalTok{Year), }\AttributeTok{format =} \StringTok{"\%Y"}\NormalTok{)}



\CommentTok{\# Import and format data on Americans foregoing medical care due to cost}
\NormalTok{forego }\OtherTok{\textless{}{-}} \FunctionTok{read\_csv}\NormalTok{(}\StringTok{"Raw Data/foregoing{-}care.csv"}\NormalTok{, }
                   \AttributeTok{col\_names =} \FunctionTok{c}\NormalTok{(}\StringTok{"Abv"}\NormalTok{, }\StringTok{"Location"}\NormalTok{, }\StringTok{"Year"}\NormalTok{, }\StringTok{"Format"}\NormalTok{, }\StringTok{"Percent"}\NormalTok{, }\StringTok{"MOE"}\NormalTok{), }\AttributeTok{skip =} \DecValTok{6}\NormalTok{)}
\end{Highlighting}
\end{Shaded}

\begin{verbatim}
## Rows: 17 Columns: 6
## -- Column specification --------------------------------------------------------
## Delimiter: ","
## chr (3): Abv, Location, Format
## dbl (3): Year, Percent, MOE
## 
## i Use `spec()` to retrieve the full column specification for this data.
## i Specify the column types or set `show_col_types = FALSE` to quiet this message.
\end{verbatim}

\begin{Shaded}
\begin{Highlighting}[]
\NormalTok{forego }\OtherTok{\textless{}{-}} \FunctionTok{subset}\NormalTok{(forego, }\AttributeTok{select =} \FunctionTok{c}\NormalTok{(}\StringTok{"Year"}\NormalTok{, }\StringTok{"Percent"}\NormalTok{))}
\NormalTok{forego}\SpecialCharTok{$}\NormalTok{Year }\OtherTok{\textless{}{-}} \FunctionTok{as.Date}\NormalTok{(}\FunctionTok{as.character}\NormalTok{(forego}\SpecialCharTok{$}\NormalTok{Year), }\AttributeTok{format =} \StringTok{"\%Y"}\NormalTok{)}



\CommentTok{\# Import and format data on ER visits per 1000 Americans}
\NormalTok{ervisits }\OtherTok{\textless{}{-}} \FunctionTok{read\_csv}\NormalTok{(}\StringTok{"Raw Data/er{-}visits.csv"}\NormalTok{, }
                     \AttributeTok{col\_names =} \FunctionTok{c}\NormalTok{(}\StringTok{"Year"}\NormalTok{, }\StringTok{"Visits"}\NormalTok{), }\AttributeTok{skip =} \DecValTok{2}\NormalTok{)}
\end{Highlighting}
\end{Shaded}

\begin{verbatim}
## Rows: 17 Columns: 2
## -- Column specification --------------------------------------------------------
## Delimiter: ","
## chr (1): Year
## dbl (1): Visits
## 
## i Use `spec()` to retrieve the full column specification for this data.
## i Specify the column types or set `show_col_types = FALSE` to quiet this message.
\end{verbatim}

\begin{Shaded}
\begin{Highlighting}[]
\NormalTok{ervisits}\SpecialCharTok{$}\NormalTok{Year }\OtherTok{\textless{}{-}} \FunctionTok{as.Date}\NormalTok{(}\FunctionTok{as.character}\NormalTok{(ervisits}\SpecialCharTok{$}\NormalTok{Year), }\AttributeTok{format =} \StringTok{"\%Y"}\NormalTok{)}



\CommentTok{\# Import and format data on assets, in thousands, of lowest 20\% of American earners}
\NormalTok{finances }\OtherTok{\textless{}{-}} \FunctionTok{read\_csv}\NormalTok{(}\StringTok{"Raw Data/finances.csv"}\NormalTok{, }
                     \AttributeTok{col\_names =} \FunctionTok{c}\NormalTok{(}\StringTok{"Year"}\NormalTok{, }\StringTok{"Status"}\NormalTok{, }\StringTok{"Income"}\NormalTok{, }\StringTok{"Assets"}\NormalTok{, }\StringTok{"Debt"}\NormalTok{), }\AttributeTok{skip =} \DecValTok{5}\NormalTok{)}
\end{Highlighting}
\end{Shaded}

\begin{verbatim}
## Rows: 7 Columns: 5
## -- Column specification --------------------------------------------------------
## Delimiter: ","
## chr (1): Status
## dbl (4): Year, Income, Assets, Debt
## 
## i Use `spec()` to retrieve the full column specification for this data.
## i Specify the column types or set `show_col_types = FALSE` to quiet this message.
\end{verbatim}

\begin{Shaded}
\begin{Highlighting}[]
\NormalTok{finances }\OtherTok{\textless{}{-}} \FunctionTok{subset}\NormalTok{(finances, }\AttributeTok{select =} \FunctionTok{c}\NormalTok{(}\StringTok{"Year"}\NormalTok{, }\StringTok{"Assets"}\NormalTok{, }\StringTok{"Debt"}\NormalTok{))}
\NormalTok{finances}\SpecialCharTok{$}\NormalTok{Year }\OtherTok{\textless{}{-}} \FunctionTok{as.Date}\NormalTok{(}\FunctionTok{as.character}\NormalTok{(finances}\SpecialCharTok{$}\NormalTok{Year), }\AttributeTok{format =} \StringTok{"\%Y"}\NormalTok{)}



\CommentTok{\# Import, truncate, and format data on health status survey data (lower value = better health)}
\NormalTok{hstatus }\OtherTok{\textless{}{-}} \FunctionTok{read\_csv}\NormalTok{(}\StringTok{"Raw Data/health{-}status.csv"}\NormalTok{,}
                    \AttributeTok{col\_names =} \FunctionTok{c}\NormalTok{(}\StringTok{"Year"}\NormalTok{, }\StringTok{"2005"}\NormalTok{, }\StringTok{"2006"}\NormalTok{, }\StringTok{"2007"}\NormalTok{, }\StringTok{"2008"}\NormalTok{, }\StringTok{"2009"}\NormalTok{, }\StringTok{"2010"}\NormalTok{, }\StringTok{"2011"}\NormalTok{, }\StringTok{"2012"}\NormalTok{, }
                                  \StringTok{"2013"}\NormalTok{, }\StringTok{"2014"}\NormalTok{, }\StringTok{"2015"}\NormalTok{, }\StringTok{"2016"}\NormalTok{, }\StringTok{"2017"}\NormalTok{, }\StringTok{"2018"}\NormalTok{, }\StringTok{"2019"}\NormalTok{), }\AttributeTok{skip =} \DecValTok{5}\NormalTok{)}
\end{Highlighting}
\end{Shaded}

\begin{verbatim}
## Rows: 7802 Columns: 221
## -- Column specification --------------------------------------------------------
## Delimiter: ","
## chr   (1): Year
## dbl  (16): 2005, 2006, 2007, 2008, 2009, 2010, 2011, 2012, 2013, 2014, 2015,...
## lgl (204): X17, X18, X19, X20, X21, X22, X23, X24, X25, X26, X27, X28, X29, ...
## 
## i Use `spec()` to retrieve the full column specification for this data.
## i Specify the column types or set `show_col_types = FALSE` to quiet this message.
\end{verbatim}

\begin{Shaded}
\begin{Highlighting}[]
\NormalTok{hstatus }\OtherTok{\textless{}{-}} \FunctionTok{as.data.frame}\NormalTok{(}\FunctionTok{t}\NormalTok{(hstatus[}\DecValTok{1}\SpecialCharTok{:}\DecValTok{22}\NormalTok{, }\DecValTok{1}\SpecialCharTok{:}\DecValTok{15}\NormalTok{]))}
\NormalTok{hstatus }\OtherTok{\textless{}{-}}\NormalTok{ hstatus }\SpecialCharTok{|\textgreater{}}  \FunctionTok{row\_to\_names}\NormalTok{(}\AttributeTok{row\_number =} \DecValTok{1}\NormalTok{)}
\NormalTok{hstatus }\OtherTok{\textless{}{-}} \FunctionTok{cbind}\NormalTok{(}\AttributeTok{Year =} \FunctionTok{rownames}\NormalTok{(hstatus), hstatus)}
\FunctionTok{rownames}\NormalTok{(hstatus) }\OtherTok{\textless{}{-}} \DecValTok{1}\SpecialCharTok{:}\FunctionTok{nrow}\NormalTok{(hstatus)}
\NormalTok{hstatus}\SpecialCharTok{$}\NormalTok{Year }\OtherTok{\textless{}{-}} \FunctionTok{as.Date}\NormalTok{(}\FunctionTok{as.character}\NormalTok{(hstatus}\SpecialCharTok{$}\NormalTok{Year), }\AttributeTok{format =} \StringTok{"\%Y"}\NormalTok{)}
\NormalTok{hstatus[,}\DecValTok{2}\SpecialCharTok{:}\DecValTok{22}\NormalTok{] }\OtherTok{=} \FunctionTok{apply}\NormalTok{(hstatus[,}\DecValTok{2}\SpecialCharTok{:}\DecValTok{22}\NormalTok{], }\DecValTok{2}\NormalTok{, }\ControlFlowTok{function}\NormalTok{(x) }\FunctionTok{as.numeric}\NormalTok{(}\FunctionTok{as.character}\NormalTok{(x)))}
\FunctionTok{colnames}\NormalTok{(hstatus) }\OtherTok{\textless{}{-}} \FunctionTok{c}\NormalTok{(}\StringTok{"Year"}\NormalTok{, }\StringTok{"All\_Ages\_adj"}\NormalTok{, }\StringTok{"All\_Ages\_crude"}\NormalTok{, }\StringTok{"Under18"}\NormalTok{, }\StringTok{"Under6"}\NormalTok{, }\StringTok{"6to17"}\NormalTok{, }\StringTok{"18to44"}\NormalTok{,}
                       \StringTok{"18to24"}\NormalTok{, }\StringTok{"25to44"}\NormalTok{, }\StringTok{"45to54"}\NormalTok{, }\StringTok{"55to64"}\NormalTok{, }\StringTok{"Over65"}\NormalTok{, }\StringTok{"65to74"}\NormalTok{, }\StringTok{"Over75"}\NormalTok{, }\StringTok{"Male"}\NormalTok{, }
                       \StringTok{"Female"}\NormalTok{, }\StringTok{"White"}\NormalTok{, }\StringTok{"Black"}\NormalTok{, }\StringTok{"Hispanic"}\NormalTok{, }\StringTok{"Under100pct"}\NormalTok{, }\StringTok{"100to199pct"}\NormalTok{, }\StringTok{"200to399pct"}\NormalTok{, }
                       \StringTok{"Over400pct"}\NormalTok{)}


\FunctionTok{summary}\NormalTok{(ervisits}\SpecialCharTok{$}\NormalTok{Visits)}
\end{Highlighting}
\end{Shaded}

\begin{verbatim}
##    Min. 1st Qu.  Median    Mean 3rd Qu.    Max. 
##   372.0   400.5   415.5   415.4   437.0   445.0
\end{verbatim}

\hypertarget{data-trends}{%
\subsection{Data Trends}\label{data-trends}}

This trend shows the average percentage of the poverty line to qualify
for Medicaid coverage. In 2013, the data leaps up to 138\% due to the
implementation of the Affordable Care Act.

\includegraphics{Overview-Markdown_files/figure-latex/pressure-1.pdf}

\end{document}
